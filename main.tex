\documentclass[a4paper,11pt]{article}
\usepackage{geometry}
\usepackage{subcaption}

\geometry{a4paper,top=3cm,bottom=3cm,left=1.5cm,right=3.5cm,%
heightrounded,bindingoffset=5mm}
%\usepackage{multirow}
\usepackage[utf8]{inputenc} %invece di \usepackage[latin1]{inputenc}
							%per caratteri accentati
\usepackage[T1]{fontenc}
\usepackage[italian]{babel}
\usepackage{graphicx}
\usepackage[svgnames]{xcolor}
\usepackage{listings}
\usepackage{hyperref}
\usepackage{mathtools}
\usepackage{algorithm}
\usepackage{pseudocode}
\usepackage[noend]{algpseudocode}
\usepackage{graphicx}
\usepackage{wrapfig}



\usepackage{fancyvrb}

\RecustomVerbatimCommand{\VerbatimInput}{VerbatimInput}
{fontsize=\tiny,
 frame=lines,  % top and bottom rule only
 framesep=1.5em, % separation between frame and text
 label=\fbox{output.txt},
 labelposition=topline,
 commandchars=\|\(\),
}


\makeatletter
\def\BState{\State\hskip-\ALG@thistlm}
\makeatother

%LOGO UNIPD
%\usepackage{eso-pic}
%\AddToShipoutPictureBG*{%
%  \AtTextUpperLeft{\makebox[\textwidth][r]{%
%    \includegraphics[width=3cm,height=2cm]{logodei.jpg}}}}

%AUTHORS
\begin{document}
\begin{itemize}
\item[ ]\textbf{Alberton Federico     matr.1106814}
\end{itemize}
\begin{center}

%TITLE
\Huge \textbf{Piano di lavoro tesi e note varie}\\ 
\normalsize{Giugno 2016}
\end{center}

\section{Obiettivi}
Andrò a modificare e testare l'algoritmo HotNet2 per poter lavorare nel caso in cui la matrice di mutazione genica non sia più essere binaria ma costituita da valori reali all'interno dell'intervallo [0,1].\\
Attualmente sono state pensate alcune modifiche da provare che verranno qui di seguito descritte brevemente come promemoria per il sottoscritto.\\
Per ora le prime idee vanno a modificare il peso che viene dato al vettore $\vec{h}$ e talvolta viene anche modificata la matrice di mutazione genica.

\subsection{L'idea Naive}
La prima idea è quella di ricondurre la matrice da reale a binaria utilizzando un soglia y in modo tale che tutti i valori al di sotto di tale soglia siano impostati a 0 ed i restanti ad 1.

\subsection{La modifica del peso}
La seconda idead è quella di modificare il peso iniziale della singola componente del vettore $\vec{h}$ ovvero per ogni singolo gene $g_i$ si va a modificare il valore $\vec{h_g}$ con la seguente formula:\bigskip\\ 
\begin{center}
$\vec{h_{g_i}} = \tfrac{\sum_{k=0}^{N-1} g_{i,k}} {|S|}$\bigskip\\
\end{center}
dove son S si intende l'insieme dei sample e con $g_{i,k}$ il valore che si trova sulla matrice genica relativo al paziente k e al gene $g_i$.

\subsection{La funzione di premio premio}
Quest'idea reutilizza il principio della soglia abbinandola ad una versione modificata dell'idea naive.\\
Come prima setto a 0 tutti i geni il cui valore non si discosta troppo dal BMR secondo una soglia $y_{BMR}$, i rimanenti geni rimarranno con il loro valore e tale valore andrà in oltre modulato tramite una funzione $P(f)$ direttamente proporzionale alla frequenza di mutazione il cui scopo è quello di premiare le mutazioni che sono più significative. Infine il premio relativo al gene $g_i$ è quello relativo alla somma dei premi relativi sempre al gene $g_i$ di tutti i pazienti che non sono stati filtrati dalla precedente soglia.\\
La funzione $P(f)$ potrebbe essere di tipo esponenziale. Altra ipotesi da valutare potrebbe essere quella di basare $P(f)$ su parametri quali la media e la varianza delle mutazioni del singolo gene sull'insieme S dei pazienti valori che sono direttamente calcolabili dai dati.











\end{document}
